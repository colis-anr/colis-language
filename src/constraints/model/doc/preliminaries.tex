
\section{Specifying POSIX Utilities in Feature Tree Logic}
%% or "Preliminaries"

\subsection{File Systems and their Models}

\paragraph{POSIX file system}
%---------
is defined by an i-node table,
        each entry of this table associates to the i-node a set of attributes,
        e.g., the type of file (ordinary, directory, link, etc.),
              the last modification time or
              a partial mapping from strings to i-nodes (directory content).

We denote by $FS$ the infinite set of file systems.

\paragraph{Feature trees}
%---------
is a model of file systems based on labeled bounded-width forests.
The labels of nodes represent attributes of a file (like its type) and
labels of edges, called features, represent atomic names used for files.
We denote by $F$ the set of features and by $D$ the set of node decorations.
Then $FT \triangleq D \times (F \rightharpoonup FT)$ denotes the set of feature
trees built with features in $F$ and decorations in $D$.
Given an element $t=(d,\sigma)\in FT$, we denote
by $\dot{t}$ the root of $t$,
by $\hat{t}$ the decoration $d$ of the root and
by $\vec{t}$ the mapping $\sigma$ at the root.

Features trees are bounded width trees.
A feature tree $t$ in $FT$ is also represented by the set of paths in the tree,
denoted by $paths(t)$, where
a path is an element of $(D\times F)^*D$
(i.e., a sequence whose elements are pairs of
        file decorations 
    and features,
ended by a decoration).

\subsection{Feature Trees Logic}

In \cite{jeannerod:hal-01807474} is defined a first order logic
to capture properties of feature trees models.
It contains the following atoms combined with classic FOL operators:
\begin{itemize}
\item $\Feat{x}{f}{y}$ for tree $x$ where the feature $f$ leads to the tree $y$;
\item $\Abs{x}{f}$ for tree $x$ with no feature $f$ (in the root);
\item $\Sim{x}{F}{y}$ for tree $x$ and $y$ having similar structure except
for features in $F$ in the root; $\Sim{x}{\emptyset}{x}$ is valid (and represents \textit{true});
\item $\Fen{x}{F}$ for $x$ contains in the root features from $F$.
\end{itemize}

The satisfiability relation $\models$ is defined by
$FT,\rho \models \varphi$
where
$\rho$ is an interpretation of free variables in $\varphi$ over features trees in $FT$.

Quantifier free FTL has a decidable satisfiability problem.
The decision procedure proposed \cite{jeannerod:hal-01807474}
proceeds by transforming the input formula to obtain an equi-satisfiable formula
in disjunctive normal form, where literals are
the one above plus $\NSim{x}{F}{y}$ and $\NFen{x}{F}$.


The set of models of a satisfiable formula $\varphi$ in QF FTL
does not have, in general, a minimal model with respect to $\prec$.
This is due to the presence of literals like
$\lnot (x\stackrel{.}{=}y)$
(i.e., $x$ denotes a tree which is not equivalent to the one denoted by $y$)
or $\NFen{x}{\emptyset}$ (i.e., $x$ does not denote an empty directory).
For example, the model generated for the last formula contains
        a fresh feature in the directory denoted by $x$;
        this feature may be different for each model built for this formula.
Figure~\ref{fig:min-model} provides an example of a satisfiable formula
which has two incomparable models.


\begin{figure}[htbp]
\begin{center}
\begin{eqnarray}
t_1 & ::= & x \stackrel{f}{\rightarrow} y \stackrel{f_1}{\rightarrow} z_1 \\
t_2 & ::= & x \stackrel{f}{\rightarrow} y \stackrel{f_2}{\rightarrow} z_2
\end{eqnarray}
\caption{Two incomparable (wrt $\prec$) models for
the formula $\Feat{x}{f}{y} \land \NFen{y}{\emptyset}$}
\label{fig:min-model}
\end{center}
\end{figure}


